% !BIB TS-program = bibtex

\documentclass[12pt,a4paper]{article}
\usepackage{amsmath}
\usepackage{graphicx}

\begin{document}

\title{DiscreteZOO documentation}
\author{Katja Ber\v{c}i\v{c} \and  Jano\v{s} Vidali}
\maketitle

\section{Citable unique identifiers}

A human-friendly shortened identifier may be obtained
by taking the first $12$ characters (hexadecimal digits) of the GUID,
splitting them into groups of $4$ characters by inserting two hyphens,
and prepending a letter \texttt{Z}.
Optionally, this letter may be followed by one or more letters
denoting the type of the object.
Finally, the name of the algorithm used to produce the identifier
may be appended after a colon at the end.
For example, we can obtain two different identifiers for the Petersen graph
from two different canonical representations:
\begin{equation}
\texttt{Zc74c-6028-a25a:bliss} \quad \text{and} \quad \texttt{ZGca5e-bcae-4138}.
\end{equation}
The first identifier specifies that it has been derived
from the canonical representation of the graph
computed using Bliss~\cite{Bliss}.
The \texttt{G} letter in the second identifier
denotes that it represents a graph;
however, the algorithm used to obtain it has not been specified.

By shortening the hash to $12$ characters out of the original $64$,
the number of such shortened identifiers drops down considerably,
yet it still remains at $2^{48} \approx 2.8 \cdot 10^{14}$.
Due to the birthday paradox, a collision
(two objects with the same identifier)
is expected to occur with probability $0.5$
when the number of identifier in the database grows to about $20$ million,
which could conceivably happen in the not so distant future.
However, we do not see it as a problem:
if this happens, the affected identifier may be extended
with additional characters from the hash for disambiguation.
Note that such an approach
is also taken by the Git versioning system~\cite{git}:
although the objects are identified by $40$-character hashes,
they are usually referred to by simply taking the first $7$ characters.

\bibliographystyle{plain}
\bibliography{docs}
\end{document}