% !BIB TS-program = bibtex

\documentclass[12pt,a4paper]{article}
\usepackage{amsmath}
\usepackage{graphicx}

\begin{document}

\title{DiscreteZOO documentation}
\author{Katja Ber\v{c}i\v{c} \and  Jano\v{s} Vidali}
\maketitle

\section{Databases}

The core database has a journaling system
that keeps track of the tables, rows and columns changed.
The reference database is stored on GitHub as a collection of JSON files.
This form is intended solely
for the ease of adding information to the database
and exporting the database into usable forms.

Each object and computed property has a corresponding list of references.
The first reference is the reference
that introduced the cell into the database.
Any other references are reserved for potential changes to the cells.

More precisely, each cell has a shadow cell with a list of changes
(commits in the versioning system)
that affected its value,
starting with the change that added the cell to the database
and followed by any changes of the value.
Furthermore, each change has a link to a reference (such as a paper).

\subsection{SageMath package}

The SageMath package works with a local SQLite database.
The user may also choose to use a PostgreSQL database,
and support for other database systems may be added easily.
The database is organised into tables representing object types,
where each column represents a single-valued property.
Multi-valued properties have their own tables
with references to the objects they represent.
The local database also has an attached journal,
which, for each change made, keeps track of the following:
\begin{itemize}
\item table changed,
\item ID of row (object or value of a multi-valued property) added or changed,
\item column (usually property) added or changed.
\end{itemize}

This makes it possible to export the changes made
for later inclusion in the core database.

\section{Citable unique identifiers}

A human-friendly shortened identifier may be obtained
by taking the first $12$ characters (hexadecimal digits) of the GUID,
splitting them into groups of $4$ characters by inserting two hyphens,
prepending the letter \texttt{Z},
and adding a plus sign and a checksum hexadecimal digit.
The checksum is obtained by taking the XOR
of the hexadecimal digits included in the citable unique identifier
(i.e., not the entire GUID).
Optionally, the initial \texttt{Z} may be followed by one or more letters
denoting the type of the object.
Finally, the name of the algorithm used to produce the identifier
may be appended after a colon at the end.
For example, we can obtain two different identifiers for the Petersen graph
from two different canonical representations:
\begin{equation}
\texttt{Zc74c-6028-a25a+8:bliss} \quad \text{and} \quad \texttt{ZGca5e-bcae-4138+0}.
\end{equation}
The first identifier specifies that it has been derived
from the canonical representation of the graph
computed using Bliss~\cite{Bliss}.
The \texttt{G} letter in the second identifier
denotes that it represents a graph;
however, the algorithm used to obtain it has not been specified.

By shortening the hash to $12$ characters out of the original $64$,
the number of such shortened identifiers drops down considerably,
yet it still remains at $2^{48} \approx 2.8 \cdot 10^{14}$.
Due to the birthday paradox, a collision
(two objects with the same identifier)
is expected to occur with probability $0.5$
when the number of identifier in the database grows to about $20$ million,
which could conceivably happen in the not so distant future.
However, we do not see it as a problem:
if this happens, the affected identifier may be extended
with additional characters from the hash for disambiguation.
Note that such an approach
is also taken by the Git versioning system~\cite{git}:
although the objects are identified by $40$-character hashes,
they are usually referred to by simply taking the first $7$ characters.

\bibliographystyle{plain}
\bibliography{docs}
\end{document}